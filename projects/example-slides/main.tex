\documentclass[xcolor=dvipsnames, aspectratio=169]{beamer}

% IMPORTS ===================================================

\usepackage{fontspec}
\usepackage{tcolorbox}

\usepackage{amsmath, amssymb, amsthm} % standard AMS packages
\usepackage{mathtools}                 % extends amsmath
\usepackage{bm}                         % bold symbols
\usepackage{physics}                    % derivatives, partials, etc.
\usepackage{esvect}                     % vector arrows
\usepackage{commath}                    % extra math macros
\usepackage{mathrsfs}                   % script letters (sigma-algebras)
\usepackage{bbm}                        % blackboard bold indicator
\usepackage{stmaryrd}                   % extra symbols
\usepackage{fixdif}

% =========================
% Graphics and diagrams
% =========================
\usepackage{tikz}              % diagrams, automata, trees
\usetikzlibrary{cd}
\usepackage{tikz-cd}
\usepackage{pgfplots}          % plots, histograms
\usepackage{forest}            % trees (binary, syntax trees)
\usepackage{qtree}             % simple trees

% =========================
% Computer Science
% =========================
\usepackage{listings}          % code listings
\usepackage{algorithm, algpseudocode} % pseudocode


% =========================
% Utilities
% =========================
\usepackage{xcolor}            % color control
\usepackage{hyperref}          % clickable references
\usepackage{cleveref}          % smarter references
\usepackage{siunitx}           % numbers and units
\usepackage{caption, subcaption} % captions for figures
\usepackage{xparse}            % flexible macros

% COLORS ====================================================
\definecolor{sqt-title}{RGB}{102, 187, 233}
\definecolor{sqt-themecolour}{RGB}{50, 183, 233}
\definecolor{sqt-question}{RGB}{102, 187, 233}

% BEAMER THEME ==============================================

\usetheme{Singapore}
\usecolortheme{rose}
\renewcommand{\thefootnote}{\textcolor{themecolour}{*}}
\beamertemplatenavigationsymbolsempty
% Make bullets black
\setbeamercolor{itemize item}{fg=black}
\setbeamercolor{itemize subitem}{fg=black}
\setbeamercolor{itemize subsubitem}{fg=black}

\setbeamercolor{structure}{fg=sqt-themecolour}
\setbeamercolor{title}{fg=sqt-title}
\setbeamercolor{subtitle}{fg=sqt-title}
\setbeamercolor{frametitle}{fg=sqt-title}

% FONTS =====================================================

\renewcommand{\footnotesize}{\scriptsize}

\usefonttheme[onlymath]{serif}
% \usefonttheme{serif}

\newfontfamily\headingfont[
  Path = style/,
  Extension = .otf
]{romela.light}
\setbeamerfont{title}{family=\headingfont}
\setbeamerfont{subtitle}{family=\headingfont}
\setbeamerfont{frametitle}{family=\headingfont}



% COLOR BOXES ===============================================
\definecolor{sqt-definition}{RGB}{255, 226, 110}
\newtcolorbox{sqt:definition}[1][]{
  colbacktitle=yellow!30,
  colback=sqt-definition,
  coltitle=black,
  title=#1,
  boxrule=0.5pt,
}

\newtcolorbox{sqt:result}[1][]{
  colbacktitle=blue!10!cyan!10,
  colback=blue!20!cyan!30,
  coltitle=black,
  title=#1,
  boxrule=0.5pt
}

\newtcolorbox{sqt:tip}[1][]{
  colbacktitle=yellow!5!green!20,
  colback=yellow!10!green!40!white,
  coltitle=black,
  title=#1,
  boxrule=0.5pt
}

% LOGO ======================================================

\logo{
  \includegraphics[height=0.8cm]{style/logo.png}
  \vspace{-0.2cm}
}

% LISTINGS ==================================================

% Define colors for syntax highlighting (Godbolt light theme)
\definecolor{codegreen}{RGB}{0, 128, 0}        % Green for comments
\definecolor{stringgreen}{RGB}{163, 21, 21}    % Dark red for strings
\definecolor{codegray}{RGB}{128, 128, 128}     % Gray for line numbers
\definecolor{codeblue}{RGB}{0, 0, 255}         % Blue for keywords
\definecolor{codepurple}{RGB}{175, 0, 219}     % Purple for types
\definecolor{backcolour}{RGB}{255, 255, 255}   % White background
\definecolor{textcolor}{RGB}{0, 0, 0}          % Black text

% Configure listings for C++ (Godbolt light style)
\lstset{
  language=C++,
  backgroundcolor=\color{backcolour},
  commentstyle=\color{codegreen}\itshape,
  keywordstyle=\color{codeblue}\bfseries,
  numberstyle=\tiny\color{codegray},
  stringstyle=\color{stringgreen},
  basicstyle=\ttfamily\footnotesize\color{textcolor},
  breakatwhitespace=false,
  breaklines=true,
  captionpos=b,
  keepspaces=true,
  showspaces=false,
  showstringspaces=false,
  showtabs=false,
  tabsize=2
}

% MINTED ====================================================

\usepackage[outputdir=build]{minted}
\usemintedstyle{vs}  % or: vs, github, colorful, etc.
\setminted{
  fontsize=\footnotesize,
  breaklines=true,
  bgcolor=white
}


\title{Making Slides}
\subtitle{Education subcommittee guide}
\institute{Sydney Quant Traders}
\date{}

\begin{document}

\maketitle

\section{Introduction}
\begin{frame}{Using tcolorboxes}
  Colored boxes are used for important definitions (yellow) and results (blue).\\
  This color-coding is used by Sheldon Axler and Zdravko Botev.

  \vspace{1cm}

  There is no requirement for the definition boxes to be self-contained definitions:
  \begin{itemize}
    \item Let $(\Omega, \mathcal H, \mathbb P)$ be a probability space
    \item Let $X:\Omega\to \mathbb R$ be $\mathcal H$-measurable
  \end{itemize}

  \begin{sqt:definition}[Definition: Expectation]
    $$\mathbb E X := \int \mathbb P(\dif \omega)\, X(\omega) $$
  \end{sqt:definition}
\end{frame}

\begin{frame}{Using tcolorboxes}
  Define the following:
  \begin{itemize}
    \item Let $(\Omega,\mathcal F, \mathbb P, \mathbb F)$ be a filtered probability space
    \item Let $f$ be a process in $L^2_{\mathbb F}$
    \item Let $I_t=\int_0^t f_s\d\!W_s$
  \end{itemize}
  \begin{sqt:result}[Theorem: It\^o isometry]
    $$\mathbb E[I_t^2] = \mathbb E\left[\int_0^t f^2_s\d s\right]$$
  \end{sqt:result}
\end{frame}

\section{Diagrams}

\begin{frame}{Images and Diagrams}
  Insert images just like you would in a normal \LaTeX\ project
  \begin{itemize}
    \item things based on pgf
    \begin{itemize}
      \item tikz
      \item pgfplots
    \end{itemize}
    \item using graphicx to insert pngs, jpegs, pdfs and others
  \end{itemize}
\end{frame}

\begin{frame}{Tikz}
  Tikz is a language used to draw diagrams in \LaTeX.
  Learning tikz is tough but you can make diagrams in 
  mathcha.io (which is a WYSIWYG), and export as tikz code.
  \begin{columns}
    \begin{column}{0.4\textwidth}
      \begin{center}
        \tikzset{every picture/.style={line width=0.75pt}}
\begin{tikzpicture}[x=0.75pt,y=0.75pt,yscale=-0.6,xscale=0.6]
\draw  [fill={rgb, 255:red, 255; green, 224; blue, 143 }  ,fill opacity=1 ] (12,119.67) .. controls (12,60.39) and (60.05,12.33) .. (119.33,12.33) .. controls (178.61,12.33) and (226.67,60.39) .. (226.67,119.67) .. controls (226.67,178.95) and (178.61,227) .. (119.33,227) .. controls (60.05,227) and (12,178.95) .. (12,119.67) -- cycle ;
\draw  [fill={rgb, 255:red, 104; green, 73; blue, 40 }  ,fill opacity=1 ] (74,153.33) .. controls (74,169.65) and (93.92,182.89) .. (118.5,182.89) .. controls (143.08,182.89) and (163,169.65) .. (163,153.33) .. controls (154.11,162.17) and (137.51,168.11) .. (118.5,168.11) .. controls (99.49,168.11) and (82.89,162.17) .. (74,153.33) -- cycle ;
\draw  [fill={rgb, 255:red, 255; green, 255; blue, 255 }  ,fill opacity=1 ] (99,167.33) -- (118.5,167.33) -- (118.5,187.96) -- (99,187.96) -- cycle ;
\draw  [fill={rgb, 255:red, 255; green, 255; blue, 255 }  ,fill opacity=1 ] (118.5,167.33) -- (138,167.33) -- (138,188.11) -- (118.5,188.11) -- cycle ;
\draw  [fill={rgb, 255:red, 139; green, 87; blue, 42 }  ,fill opacity=1 ] (77,113.17) .. controls (77,103.13) and (81.7,95) .. (87.5,95) .. controls (93.3,95) and (98,103.13) .. (98,113.17) .. controls (98,123.2) and (93.3,131.33) .. (87.5,131.33) .. controls (81.7,131.33) and (77,123.2) .. (77,113.17) -- cycle ;
\draw  [fill={rgb, 255:red, 139; green, 87; blue, 42 }  ,fill opacity=1 ] (140,113.17) .. controls (140,103.13) and (144.7,95) .. (150.5,95) .. controls (156.3,95) and (161,103.13) .. (161,113.17) .. controls (161,123.2) and (156.3,131.33) .. (150.5,131.33) .. controls (144.7,131.33) and (140,123.2) .. (140,113.17) -- cycle ;
\draw  [line width=4.5]  (38,92.33) .. controls (43,86.33) and (52,85.33) .. (74,84.33) .. controls (96,83.33) and (102,84.33) .. (109,93.33) .. controls (116,102.33) and (114,111.33) .. (110,122.33) .. controls (106,133.33) and (99,139.33) .. (78,139.33) .. controls (57,139.33) and (51,137.33) .. (47,128.33) .. controls (43,119.33) and (33,98.33) .. (38,92.33) -- cycle ;
\draw  [line width=4.5]  (129,88.33) .. controls (137,84.33) and (137,84.33) .. (161,83.33) .. controls (185,82.33) and (193,84.33) .. (197,92.33) .. controls (201,100.33) and (193,115.33) .. (189,124.33) .. controls (185,133.33) and (175,137.33) .. (158,138.33) .. controls (141,139.33) and (134,134.33) .. (127,123.33) .. controls (120,112.33) and (121,92.33) .. (129,88.33) -- cycle ;
\draw [line width=5.25]    (113,104.33) -- (123,104.33) ;
\draw [line width=5.25]    (29,86.33) -- (38,92.33) ;
\draw [line width=5.25]    (197,92.33) -- (206,86.33) ;
\end{tikzpicture}
      \end{center}
    \end{column}
  
    \begin{column}{0.6\textwidth}
      \begin{sqt:definition}[Definition: Nerd]
        Yes I am a nerd, bookworm, I'm studious.
        From my cerebral cortex to my gluteus.
      \end{sqt:definition}
    \end{column}
  \end{columns}

\end{frame}

% here ---------\
%               v
\begin{frame}[fragile]{Beamer Fragile Issues}
  \begin{sqt:result}[Theorem: Beamer Fragile Issues]
    Sometimes your \LaTeX\ code works outside of a Beamer frame, but doesn't work
    when put inside a Breamer frame.
    I have no idea how this happens. But if you add the optional argument ``fragile''
    next to the \verb|\begin{frame}| statement, then it works.
  \end{sqt:result}

  \begin{center}
    \begin{tikzcd}
      U\cap V \rar[hook]{i}\dar[hook]{j} & 
      U \dar[hook]{j'}\arrow[ddr, bend left, "\forall f"] &\\
      V \arrow[r, hook, "i'"]\arrow[drr, bend right, "\forall g"'] & 
      X \arrow[dr, dotted, "\exists ! h" description] &\\
      & & \forall Y
    \end{tikzcd}
  \end{center}
\end{frame}

\begin{frame}{Drawing Diagrams}
  You may use anything you want to draw diagrams,
  as long as the result looks clean and professional.
  I highly recommend Inkscape over tikz, for a fast workflow.

  \centering
  \includegraphics[width=5cm]{assets/example.pdf}

\end{frame}

\begin{frame}[fragile]{Matplotlib}
  \centering
  \includegraphics[width=10cm]{assets/brownian_motion.pdf}
\end{frame}

\begin{frame}[fragile]{Matplotlib}
  \begin{minted}{py}
    # Configure matplotlib to use PGF backend for LaTeX compatibility
    matplotlib.use("pgf")
    matplotlib.rcParams.update({
        "pgf.texsystem": "pdflatex",
        "font.family": "serif",
        "font.size": 11,
        "text.usetex": True,
        "pgf.rcfonts": False,
    })
    # ... make plot ...

    # Save as PGF file
    plt.savefig(save_path, bbox_inches='tight')
    print(f"Plot saved to {save_path}")
    
    # Also save as PDF for preview
    pdf_path = save_path.replace('.pgf', '.pdf')
    plt.savefig(pdf_path, bbox_inches='tight')
    print(f"PDF preview saved to {pdf_path}")
  \end{minted}

\end{frame}

\section{Coding}

\begin{frame}[fragile]{Code Snippets}

  You can get proper syntax highlighting with \verb|minted|.\\
  This requires having python and Pygments installed on your machine.\\
  Code snippets won't work inside a Beamer frame unless you use \verb|fragile|.

  \begin{minted}{cpp}
    template<class T, size_t N = 10>
    class Buffer : public Container<T> {
        T* data_{new T[N]};  // meow
    public:
        ~Buffer() { delete[] data_; } // woof
        T& operator[](size_t i) { return data_[i]; }
    };
  \end{minted}

  Notice that all indentation in the \LaTeX\ source code
  within the snippet is shown verbatim in the rendered slide. 
\end{frame}

\begin{frame}[fragile]{More Code Examples}
  \begin{columns}
  \begin{column}{0.5\textwidth}
  \begin{minted}{cpp}
    template<typename T>
    class Stack {
    private:
        std::vector<T> data;
        
    public:
        void push(const T& item) {
            data.push_back(item);
        }
        
        T pop() {
            T value = data.back();
            data.pop_back();
            return value;
        }
        
        bool empty() const; // omitted
    };
  \end{minted}
  \end{column}
  \begin{column}{0.5\textwidth}
    You can put diagrams here to assist with 
    explaining the code.

    \begin{center}
      \includegraphics[width=4cm]{assets/stack.png}
    \end{center}
  \end{column}
\end{columns}
\end{frame}


\section{Further Examples}

      



\end{document}